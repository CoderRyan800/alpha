\documentclass{article}

\title{Simple Reasoning and Knowledge States in a LSTM-Based Agent}

\author{Ryan Mukai}

\date{14 December 2018}

\begin{document}

\maketitle

\section{Introduction}

This paper focuses on solving a simple propositional logic
problem using LSTM neural network based agents.
We present an agent capable of displaying its knowledge
state and of answering questions based on its state of knowledge.  If an
agent is unable to answer a question, it will indicate
this and request assistance from another agent.  The other agent,
upon receiving such a request, provides data from its
knowledge state to aid the requestor in its goal of finding an answer.

Much work has occurred the field of neural networks and symbolic reasoning.
A good survey of the field, with numerous references to recent developments, is
\cite{DBLP:journals/corr/abs-1711-03902}, which is also a nice introduction to
the field.
A good but older introduction to neural-symbolic learning is \cite{neural_symbolic_2002},
and another overview of the field is in
\cite{bica2015_integrated_neural_symbolic}.
Reference
\cite{neural_symbolic_2007} describes many approaches used in neuro-symbolic
learning systems, including the SHRUTI spiking neural network
\cite{shruti_2007}, the Core Method for learning first order logic programs
\cite{core_method_2007}, topos theory for learning models of
predicate logical theories \cite{topos_theory_2007}, modal and temporal
reasoning with neural networks \cite{modal_temporal_2007}, and multi-valued
logic and neural networks \cite{multi_valued_logic_2007}, and many other
approaches have been described in the literature.  The use of LSTMs for
neuro-symbolic tasks also has a precedent,
and one example is \cite{captcha_2017}.

The focus of this paper differs
in that we have LSTM agents base their actions on their state of knowledge and
cooperate to solve a problem.  We do not, however, pursue a direct relationship
between neural connections, weights, and symbolic representations as in most
of the literature of the field
\cite{DBLP:journals/corr/abs-1711-03902}.  Also, we are not aiming for logical
completeness and are instead focusing on the following topics:
\begin{enumerate}
\item Use of LSTMs for simple logical reasoning, motivated by LSTM memory
capabilities and by desire to possibly apply this to chatbots which maintain
state and interact with users.
\item Having LSTMs maintain a limited concept of a logical sentence as a unit
to be stored in memory for later recall.
\item Having LSTMs have a limited concept of their own knowledge in the sense
of being able to dump their knowledge state on request.
\item Having LSTMs have a limited concept of their own knowledge in the sense
of being aware of not being able to answer a question and asking for help.
\item Having a simple scenario of LSTM agents that can cooperate and share knowledge
to solve a simple problem.
\end{enumerate}


We create two simple LSTM-based agents to solve propositional
logic problems involving simple syllogisms.  Each agent is given a
simple knowledge base and is trained to respond to questions based on
its available knowledge.  If it lacks an answer, the agent is trained to
ask for help.  Each agent is also trained to respond to a request for
help by simply dumping its knowledge base, possibly yielding clues to
help a requestor in finding an answer.  An agent that is able to answer
a question does not request help.

In this paper, all reasoning involves propositional logic.

\section{The Problem Agents are Trained to Solve}

Syllogisms are a simple and well known form in logical reasoning,
and we train agents to handle simple problems of the following forms.

\begin{enumerate}
\item Given the status of a propositional variable $a$ as
true, false, or unknown, respond to a question about its value.
\item Given $a \rightarrow b$ and given $a$ is true, answer
that $b$ is true if asked about $b$.
\item Given $a \rightarrow b$ and given $b$ is false, answer that
$a$ is false if asked about $a$.
\item Given the value of $a$ and no other information, if asked about
$b$ indicate $b$ is unknown.
\item Given the statement $a \rightarrow b$ and nothing else, if asked
about either $a$ or $b$, conclude the requested value is unknown.
\item Given no statements at all or given only statements
that say nothing about $a$, if asked about $a$, conclude $a$ is
unknown.
\end{enumerate}

The above involves simple propositional reasoning about simple syllogisms.
In addition, we also want an agent to exhibit these behaviors.

\begin{enumerate}
\item If the answer to a question is unknown, then ask another agent for
help.
\item If the answer to a question is known, given the correct answer and
do not ask for any help, since help is not needed.
\item If a request for help is received, dump the statements in one's
knowledge base.  This may aid the requestor in answering the
original question.
\end{enumerate}

The second list of items above constitutes a very simple form of knowledge
of internal state. Here, an agent should be aware of whether it knows the
right answer, which is either true or false, or whether it needs help
because the answer
is unknown.  An agent should also be aware of the statements in its
knowledge base and be capable of supplying them when asked for help.

\section{The Agent}
\label{the_agent}
The core of the agent consists of an LSTM neural network created
using Keras \cite{chollet2015keras}
with Tensorflow \cite{tensorflow2015-whitepaper}
as a backend.  The network was created using
the Keras functional API as follows.

\begin{itemize}
\item
  \begin{verbatim}
    x1 = LSTM(256, return_sequences=True)(inputs)
  \end{verbatim}
\item
  \begin{verbatim}
    x2 = LSTM(256, return_sequences=True)(x1)
  \end{verbatim}
\item
  \begin{verbatim}
    x3 = (Dense(Y1[0,0].size))(x1)
  \end{verbatim}
\item
  \begin{verbatim}
    y1 = Activation('softmax')(x3)
  \end{verbatim}
\item
  \begin{verbatim}
    x4 = (Dense(Y1[0,0].size))(x2)
  \end{verbatim}
\item
  \begin{verbatim}
    y2 = Activation('softmax')(x4)
  \end{verbatim}
\item
  \begin{verbatim}
    model = Model(inputs=inputs, outputs=[y1, y2])
  \end{verbatim}
\end{itemize}

The inputs are logic sentences represented with one-hot encoding.
These may contain repeated sentences.  These can also contain
obsolete unknown statements.  We use an example to illustrate.
Suppose the list of sentences below, one-hot encoded, is our input.

\begin{enumerate}
\item $a$ is unknown
\item $b$ is unknown
\item $a$ is unknown
\item $a \rightarrow b$
\item $a$ is true
\item $b$ is unknown
\end{enumerate}

The output y1 consists of the following sentences, represented
using one-hot encoding.

\begin{enumerate}
\item $b$ is unknown
\item $a \rightarrow b$
\item $a$ is true
\end{enumerate}

Here we note that the sentence ``$a$ is unknown'' represents
an obsolete knowledge state that is replaced by ``$a$ is true''.
Note also that the agent still has ``$b$ is unknown'' stored as
it does not, at this stage, conclude that $b$ is true.  For the
purposes of $y1$ it is merely reciting the sentences in the order
in which they were learned with obsolete unknowns removed and in the
same order of learning.  Note that repetitions are purged.  Hence,
the purpose of training the first of two LSTM layers and a corresponding
Dense output layer on y1 is to teach the system to develop both a
concept of replacing old unknowns with new knowledge and, importantly,
to teach the system the concept of sentences as units of information or
knowledge.

No matter what question is asked, the output $y1$ is always a
simple recital of knowledge sentences, without drawing any logical
conclusions.

The output $y2$ is, however, dependent on the question asked.

\begin{itemize}
\item Question ``what is $a$ ?'' results in ``a is true''
\item Question ``what is $b$ ?'' results in ``b is true''
\item Question ``what is $c$ ?'' results in ``c is unknown . help''
\item Question ``help'' results in ``$b$ is unknown . $a \rightarrow b$ . $a$ is true''
\end{itemize}

Note that the response to ``help'' makes $y2$ the same as $y1$.  Here,
$y1$ can be thought of as an ``internal'' or ``private'' output.  By
contrast, $y2$ can be thought of as the ``external'' or ``public'' output.

The simulation environment is designed to only have an
output if the network does not have the keyword ``help'' in its output.
But if the word ``help'' appears, then the environment sets a help flag
and send the question ``help'' to another agent.  The other agent's response,
the dump of its knowledge base, is sent back to the first agent, and the first
agent tries to answer the question again.

The present system assumes that all knowledge presented to it
is logically consistent.  Although the statement ``a is unknown'' can be
replaced with the statement ``a is true'' or with the statement
``a is false'', the present networks are not trained to handle situations
in which a previous belief is reversed.  In particular, they are not
trained for the case in which a previous belief such as ``a is true''
is replaced by a belief that ``a is false'' or vice versa.  There
exist systems that are able to process such updates in knowledge
but that is outside of the scope of the present work, which focuses
on teaching LSTMs to update their knowledge assuming no inconsistencies
and is focused on awareness of knowledge state and on cooperation between
two agents.

\subsection{Agent Cooperation: Example 1}
\label{example_1}

Suppose we have two agents, Agent 1 and Agent 2.
Agent 1 is given the following knowledge:

\begin{itemize}
\item $a \rightarrow b$
\end{itemize}

Agent 2 is given the following knowledge:

\begin{itemize}
\item $a$ is true
\end{itemize}

If we ask Agent 1 ``what is $b$ ?'', the following will occur.

\begin{enumerate}
\item Agent 1 indicates $b$ is unknown and requests help.
\item The simulation environment, in response to seeing Agent 1
request help, send a help query to Agent 2.
\item Agent 2 dumps its knowledge that $a$ is true.
\item This knowledge is added to the knowledge base of Agent 1.
\item Agent 1 runs again and is able to finally conclude $b$ is true.
\end{enumerate}

In this example, we note the following:

\begin{enumerate}
\item Agent 1 lacks adequate information to determine $b$ and
indicates a lack of knowledge of $b$.
\item Agent 1 will, without knowledge of $b$, make a request for help.
\item Agent 2 responds to a help query by dumping its knowledge state.
\item This allows Agent 1 to find the answer.
\end{enumerate}

Agents only request help if they do not know the answer but will otherwise
answer the question.  This implies that an agent has a limited knowledge of
its own knowledge state.  If one knows the answer, one answers the question.
If one does not know an answer, one indicates this is so and requests help.

Hence,
\begin{enumerate}
\item Agents have a limited concept of knowing whether or not they know
an answer and requesting help when they do not.
\item Agents also have knowledge
states that they can dump if they are asked for help.
\item Moreover, in terms of maintaining an internal knowledge state, an agent
knows how to purge obsolete known sentences of the form ``$a$ is unknown''
with ``$a$ is true'' or with  ``$a$ is false'', implying the ability to
understand that once one knows something is true or false then it is not
unknown any more.
\end{enumerate}

\subsection{A Second Example}
\label{example_2}
Agent 1 is given the following knowledge:

\begin{itemize}
\item $a \rightarrow b$
\end{itemize}

Agent 2 is given the following knowledge:

\begin{itemize}
\item $b$ is false
\end{itemize}

If we ask Agent 1 ``what is $a$ ?'', the following will occur.

\begin{enumerate}
\item Agent 1 indicates $a$ is unknown and requests help.
\item The simulation environment, in response to seeing Agent 1
request help, send a help query to Agent 2.
\item Agent 2 dumps its knowledge that $b$ is false.
\item This knowledge is added to the knowledge base of Agent 1.
\item Agent 1 runs again and is able to finally conclude $a$ is false.
\end{enumerate}

In the next Section, we describe the training templates used in this
simple agent.


\section{Training the Agent}

At the heart of the agent we have a deep learning neural network with
LSTM and Dense layers as described in Section \ref{the_agent}.  A network
must be trained to respond to questions that are presented to it given a
certain knowledge base.

Each training template contains a list of statements, a question, and an
answer.  Here is an example:

\begin{verbatim}

training_template = {
  'statement_list' : ['if pa1 then pa2', 'pa2 is false'],
  'question' : 'what is pa1 ?',
  'answer' : 'pa1 is false'
}

\end{verbatim}

In the above example, ``pa1'' and ``pa2'' are placeholders.  When the
training examples are generated, they will be replaced by actual variables
from the set ${a0, a1, \ldots, a9}$ in order to create a training example, so
there are 90 different variable combinations possible for the template above.
This is necessary in order to teach the network to generalize well.  Not only
are different variables used in the placeholders when creating problems from
a template, but the input sentences in the ``statement\_list'' value will
be presented in random order, with possible repetitions and with each
statement being presented one or more times.

\begin{itemize}
\item The resulting input ``X'' for a problem instances will be a one-hot encoded
set of statements based on the statement list as described above.  It will
include the statements from the statement list presented in random order with
repetitions and end with the ``question''.
\item The first output ``Y1'' will be a version of the ``X'' in which any
each statement is presented only once and in the order in which it appeared
in ``X''.  Hence, ``Y1'' is the knowledge state, and we are teaching the
network to output its knowledge state in ``Y1'' by repeating sentences in
the order they were learned.  This constitutes a knowledge memory.
\item The second output ``Y2'' is a one-hot encoded version of the ``answer''
to the ``question''.
\end{itemize}

Since we can generate multiple problem instances for each template,
each template also contains a ``max\_instances'' member as well, which tells
the system the maximum number of randomly generated problems to create
for the given template.
Different placeholders (i.e. ``pa1'' and ``pa2'') also correspond to distinct
variables but that the same variable is always used for a given placeholder
during the instantiation process.

A full list of the training templates is shown the Github repository
\cite{github_repo}.

\section{Key Results}

The network was found to yield no errors on the validation data set,
consisting of 5000 sample problems, after being trained through over 1700
epochs of
training data.  It was subsequently also tested as in examples 1 and 2
in Sections \ref{example_1} and \ref{example_2} as well.

\section{Some Limitations}

It must be noted that in the present implementation, an Agent is a Python
object, and the knowledge base is still maintain as a Python list of
sentences that is presented to a newly initialized LSTM on each call to
the agent.  Each run of the neural network involves presenting the
sentences and the question (where ``help'' is one possible question)
to the network and subsequently obtaining its output.  Maintenance of LSTM
state across calls could be used in future editions of the Agent to eliminate the need
for maintaining an object with a list of sentences.

At present, agents are not trained to
handle more complex reasoning chains such as $a$, $a \rightarrow b$,
$b \rightarrow c$, and then conclude $c$ is true.  This would imply there
is still an incompleteness in agent knowledge in the sense that a more complete
agent would have the power to first conclude the truth of $b$ and combine
that knowledge via modus ponens with $b \rightarrow c$ to conclude $c$.
Also, the agents are not trained to solve general propositional reasoning
problems and can only handle the limited task scope described above.
This is considered a direction for future work.

\section{Summary}

A deep learning neural network using LSTM and Dense layers with the ability
to perform a very limited symbolic propositional reasoning task has been created
and demonstrated.  Agents with the ability to answer a question if the answer
is known and with the ability to recognize a lack of knowledge and ask for
help have been built, and these agents can also respond to requests for help
with a knowledge state dump.  The concept of learning a knowledge state
has been demonstrated in a simple case.

\section{Appendix}

Source code is available at https://github.com/CoderRyan800/alpha.git and
can be installed as follows.

\begin{itemize}
\item Create a Python 3.6 or Python 3.7 virtual environment.
\item Activate the new virtual environment.
\item ``git clone https://github.com/CoderRyan800/alpha.git''
\item ``cd alpha''
\item ``pip install .''
\item ``cd neural''
\end{itemize}

The main program is nn\_entity\_v1\.py.  On line 544 of the code you can
set the flag new\_network\_flag to False if you wish to use the pre-built
neural network located in the data subdirectory.  You can set it to True
if you want to train the neural network yourself, but unless you have a GPU
it may take many days to train.

The Github repository was called ``alpha'' because the author regards this
as alphaware.  The author does not claim suitability for any purpose and
only provides the software ``as is''.

\emph{IMPORTANT}: If you do not choose to train a new network you have to
download a pre-existing data set.  You can go to:
\begin{verbatim}
http://alpha-demo.s3-website-us-east-1.amazonaws.com/
\end{verbatim}
This is
needed not only for the data set but also to obtain the neural network
trained after over 1700 iterations if you wish to avoid retraining.
Put these downloaded files into the ``data'' subdirectory in the
alpha repository on your local system prior to trying to run with the
flag on line 544 set to False.

\bibliographystyle{plain}

\bibliography{biblio_v2}

\end{document}
